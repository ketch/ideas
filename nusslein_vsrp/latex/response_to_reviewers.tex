\documentclass[11pt]{letter}
\usepackage[USenglish]{babel}
\usepackage[utf8]{inputenc}
\usepackage[a4paper]{geometry}
\usepackage[plainpages=false,pdfpagelabels,hidelinks,unicode]{hyperref}
\hypersetup{pdftitle={Response to Reviewers}}
\usepackage{enumitem}

\usepackage{xr} % for references to another document
\externaldocument{overview}

\usepackage{color}
\newcommand{\todo}[1]{{\Large \textcolor{red}{#1}}}

\newcommand{\revA}[1]{{\color{red}#1}}
\newcommand{\revB}[1]{{\color{blue}#1}}
\newcommand{\revC}[1]{{\color{cyan}#1}}

\usepackage{amssymb}
\usepackage{commath}

\usepackage{xspace}
\newcommand{\cf}[0]{{cf.\@}\xspace}
\newcommand{\eg}[0]{{e.g.\@}\xspace}
\newcommand{\ie}[0]{{i.e.\@}\xspace}
\newcommand{\etc}[0]{{etc.\@}\xspace}

% fonts
\usepackage{ifluatex}
\ifluatex
  \usepackage[no-math]{fontspec}
\else
  \usepackage[T1]{fontenc}
\fi
\usepackage{newpxtext,newpxmath}


% \date{June 16, 2020} %TODO: date


\begin{document}
\begin{letter}{}

\textbf{Response to Reviewer}

\opening{Dear Editor, Dear Reviewer,}

we are very grateful for the thorough review and helpful comments on our manuscript
``Positivity-Preserving Adaptive Runge--Kutta Methods''.
We revised the manuscript in accordance with the reviewer's suggestions.
Please find below a point-by-point description of these changes and reply to
the reviewer's comments.



\textbf{\large Reviewer \#1 \revA{(manuscript changes highlighted in red)}}
% TODO: Reviewer 1

\emph{%
This manuscript studies the efficiency of a new kind of Positivity-Preserving
Runge-Kutta Methods. The main idea in the new approach consists in adaptively
choosing the weights b of the RKM in a way that ensures positivity, or other
bound constraints, for problems satisfying the sufficient condition (1.4).
One of the main benefits of this new approach is that all linear invariants of
the problem are preserved. Other projection methods also preserve invariants,
but only if you explicitly know these invariants. If any component of the
numerical solution is negative, a linear program (LP) must be solved at
each step in order to choose the new weights, and this could be an extra time
consuming process. Nevertheless, the idea of minimizing the deviation of the
weights is simple and in some cases may be much more efficient, particularly
for large systems. The numerical experiments are well presented and quite
thorough including different stiff and non-stiff problems, with fixed and
adaptive step size implementations.
\\
Positivity-Preserving methods have been studied quite a lot in recent decades
and the authors have summarized really well the state of art in the introduction
by referring to Projection methods, SSPRK methods, MPRK methods and DSRK positive
methods. In the new schemes proposed the weights of some baseline RKM are adapted
in order to ensure positivity or other bounds. These new methods are valid both
for stiff and non-stiff problems satisfying the sufficient condition for
positivity (1.4). In addition, the authors clarify that for some positive
problems not satisfying (1.4) numerical tests did not show promising results.
I personally value the algorithm given in Section 3. Theorem 4.1 about the
stability region of the new adaptive Runge-Kutta method with convex adaptation
is quite interesting, particularly for A-stable methods. Lemma 4.2 with the
bound for the stability function of the adapted method is interesting too. The
paper is clear and well written, thus I would recommend this work to be published
provided that the authors address properly the points listed below.
}

\begin{enumerate}[label=\arabic*.]
  \setcounter{enumi}{1}
  \item Bound-preserving adaptive Runge–Kutta methods
  \begin{enumerate}[label=\alph*)]
    \item \emph{%
    In equation (2.14b), it is weird to read $-0.01 u_2$ and $- 0.05 u_2$
    in the same line. Is it an error? The same thing happens in equations (2.14c)
    with $u_3$, in (5.5b) with $u_2$, and in (5.5c) with $u_3$.
    }

    We added an explanation why we use this seemingly weird notation of the
    equations [lines~\ref{line:prod-dest}ff.].


    \item \emph{%
    In Figure 1, in the difference $\| \tilde b - b \|$, a tilde is missing.
    }

    We added the tilde to the plot.
  \end{enumerate}


  \item Selection of the modified weights
  \begin{enumerate}[label=\alph*)]
    \item \emph{%
    In section 3.2, two different approaches to adapt the weights are considered:
    Free adaptation and Convex adaptation. At the same time, it is explained that
    ``...we can additionally impose either or both of the following strategies \dots''.
    I think that the first strategy is nothing but the second approach (Convex
    adaptation). I suggest making this part clearer in order to avoid confusion
    between approaches and strategies.
    }

    We modified this part to be more clear [lines~\ref{line:strategies}ff.].
  \end{enumerate}


  \item Properties of adaptive RKMs
  \begin{enumerate}[label=\alph*)]
    \item \emph{%
    Figure 3a, with free adaptation of the weights $\tilde b$, does not give enough
    information about the stability regions of the adapted methods. In my opinion,
    it would be clearer if each stability curve was labeled with the value of
    the difference $\| \tilde b - b \|$, accordingly with the values in Figure 4b.
    In this way we could see the progress of the stability regions during the
    numerical integration process.
    }

    \revA{TODO: Review the following:} %TODO
    We found an additional issue: The step size used for this plot was not $\Delta t = 0.015$ as for Figure 4a. We corrected the step size. Therefore the shape of the plotted stability regions changed. 
	To indicate the value of $\| \tilde b - b \|$ for the different stability regions the boundaries were colored using a colormap.  
	We removed the filling pattern to improve clarity.
    


    \item \emph{%
    In the first line of formula (4.6) the vector $(\tilde b - b)$ should be
     $(\tilde b - b)^T$.
    }

    We fixed equation \eqref{eq:proof-of-lemma-free-adaptation}.
  \end{enumerate}


  \item Numerical experiments
  \begin{enumerate}[label=\alph*)]
    \item \emph{%
    In the third line of Section 5.1, RHS appears for the first time without
    explaining its meaning.
    }

    We changed this sentence and introduced the abbreviation RHS
    [line~\ref{line:RHS}].


    \item \emph{%
    In section 5.1, just before the formula (5.1), it is said ``Since the
    positivity of the stage values is not ensured, the right hand side has to be
    defined for negative values. If this is not the case, the right hand side
    has to to be adapted accordingly''. This is a bit confusing, particularly
    the text ``...has to be defined for negative values...''. I suggest that
    you make this part clearer.
    }

    We tried to clarify this part by giving an example and adding further
    explanations [lines~\ref{line:RHS-negative-values}ff.].


    \item \emph{%
    In formula (5.2), the last $u_{i-1}$, shouldn’t it be $u_i$?
    }

    We fixed equation \eqref{eq:advection-decay-semidiscrete}.


    \item \emph{%
    After formula (5.2) ``...In Figure 4a the results for $\Delta t = 0.015$
    are plotted for different time steps...'' should be ``...In Figure 4a the
    results for $\Delta t = 0.015$ are plotted for different values of time $t$...''
    }

    We fixed this sentence [line~\ref{line:time-vs-dt}].


    \item \emph{%
    Figure 5, about the non-stiff problem with fixed stepsize, is quite interesting
    and gives information about the convergence of the adapted mehtod. For
    $\Delta t = 0.0082$ the unaltered method leads to positive solutions. For
    $\Delta t = 0.015$ the original method leads to negative values in the
    interval $[0, 0.375]$ and the weights are altered by solving an LP problem.
    Is this extra time consuming a significant cost or could it be an economical
    alternative to using excessively small step sizes? A further explanation
    about this issue should be added.
    }

    \revA{TODO: Stephan wants to run some numerical experiments} %TODO


    \item \emph{%
    In Figures 6c and 6d, in the difference $\| \tilde b - b \|$ and also in the
    legend, a tilde is missing.
    }

    We added the tilde to the plot.


    \item \emph{%
    In the third subplot of Figure 9, in the difference $\| \tilde b - b \|$,
    a tilde is missing.
    }

    We added the tilde to the plot.


    \item \emph{%
    In the fourth subplot of Figure 11, in the difference $\| \tilde b - b \|$,
    a tilde is missing.
    }

    We added the tilde to the plot.
  \end{enumerate}
\end{enumerate}








\end{letter}
\end{document}
