\documentclass{article}

\usepackage{amsmath}
\usepackage{amssymb,amsfonts}
\usepackage[hmargin=0.9in]{geometry}

\usepackage{subfig}
\usepackage{graphicx}
\usepackage[colorlinks=true,citecolor=blue,urlcolor=blue]{hyperref}
\usepackage{cite}


\newtheorem{remark}{Remark}
\numberwithin{remark}{subsection}

\newcommand{\bfv}{{\bf v}}
\newcommand{\bfn}{{\bf n}}
\newcommand{\bfx}{{\bf x}}

\title{Ideas for using Runge--Kutta methods to preserve positivity and similar properties}


\begin{document}
\maketitle

\section{Basic relaxation approach}

\section{Linear programming approach}
Given a Runge--Kutta method with coefficients $(A,b)$, after we have computed
the stages
$$
    y_j = u^n + h \sum_i a_{ij} f(y_i),
$$
the idea is to adaptively choose the weight $b_j$ in order to enforce positivity.
This leads to a linear program with the linear inequality constraints
$$
u^{n+1}_i = u^n_i + h \sum_j b_j f(y_j) \ge 0
$$
and the linear equality constraints that come from the order conditions.  The idea is
to enforce as many of these as possible.  For first order:
\begin{align}
\sum_j b_j = 1.
\end{align}
For second order:
\begin{align}
\sum_j c_j b_j = 1/2.
\end{align}
For third order:
\begin{align}
\sum_j c_j^2 b_j & = 1/3 \\
\sum_j \tau_{2j} b_j & = 0.
\end{align}
For fourth order:
\begin{align}
\sum_j c_j^3 b_j & = 1/4 \\
\sum_j \tau_{2j} c_j b_j & = 0 \\
\tau_{2}^T A^T b & = 0 \\
\tau_{3}^T b & = 0.
\end{align}
For fifth order:
\begin{align}
\tau_2^T A^T C b & = 0 \\
\tau_2^T C A^T b & = 0 \\
\tau_2^T (A^T)^2 b & = 0 \\
\tau_2^T C^2 b & = 0 \\
\tau_2^2 b & = 0 \\
\tau_3^T A^T b & = 0 \\
\tau_3^T C b & = 0 \\
(\tau_2^T)^2 b & = 0 \\
\tau_4^T b & = 0.
\end{align}
Here vector powers mean componentwise multiplication while
matrix powers mean matrix multiplication.  The matrix $C$ is the
diagonal matrix of the abscissas $c_i$ and
$\tau_k$ are the stage order residual vectors:
$$
    \tau_k = \frac{1}{k!} c^k - \frac{1}{(k-1)!}Ac^{k-1}.
$$
Since they only appear in equations where the right-hand-side is zero,
it's okay to multiply by $k!$ and use instead the vectors
$$
    \hat{\tau}_k = c^k - kAc^{k-1}.
$$

\clearpage
\section{Runge-Kutta methods with flux limiting}
Consider the transport equation
\begin{align}\label{transport_equation}
  \partial_t u + \bfv\cdot \nabla u = 0
\end{align}

\subsection{Forward Euler with low-order discrete upwinding}\label{sec:FE_discUpwind}

Let $m_i=\int_\Omega \phi_i dx$ be the $i$-th entry in the diagonal lumped mass matrix
and assume $m_i>0$.
Then, following \cite{kuzmin2012flux},
the full discretization with discrete upwinding is given by
\begin{align}\label{low-order}
  m_i\left(\frac{U_i^L-U_i^n}{\Delta t}\right)+\sum_j\left(K_{ij}+D_{ij}^L\right)U_j^n=0,
\end{align}
where $U^L$ is (the vector of DOFs of) the low-order solution at time $t^{n+1}$,
$U^n$ is the solution at time $t^n$ and 
\begin{align}\label{transport_and_discUpwinding_matrices}
  K_{ij} = \int_\Omega (\bfv\cdot\nabla \phi_j) \phi_i dx,
  & &
  D^L_{ij} =
  \begin{cases}
    -\max(K_{ij},0,K_{ji}), & \text{ if } i\neq j, \\
    -\sum_{j\neq i} D_{ij}^L, & \text{ if } i=j
  \end{cases}
\end{align}
are the transport and artificial dissipative matrices respectively.

\begin{remark}[About the discrete maximum principle (DMP).]
  The solution $U^L$ satisfies the following local bounds:
  \begin{align}\label{local-bounds}
    U_i^{\min}:= \min_{j\in N_i} U_j^n \leq U_i^L \leq \max_{j\in N_i} U_j^n=:U_i^{\max},
  \end{align}
  where $N_i$ is the set of DOFs that define the stencil of the discretization at the $i$-th node.
  Rewrite equation \eqref{low-order} as $U_i^L=\sum_j R_{ij}U_j^n$ with
  $R_{ij}:=\delta_{ij}-\Delta t m_i^{-1} (K_{ij}+D_{ij}^L)$, where $\delta_{ij}$ is
  the Kronecker delta function. Note that by construction
  $K_{ij}+D^L_{ij}\leq 0, \forall i\neq j\implies R_{ij}\geq 0, \forall i\neq j$.
  The diagonal entries of $R$ are positive provided $\Delta t$ is small
  enough such that $1-\Delta t m_i^{-1}(K_{ii}+D_{ii}^L)>0, \forall i$. 
  Note also that, due to partition of unity and by construction of the dissipative operator $D^L$,
  $\sum_j K_{ij}+D^L_{ij}=0\implies \sum_j R_{ij}=1$. Therefore,
  \begin{align*}
    U_i^L=\sum_j R_{ij}U_j^n\leq U_j^{\max}\sum_j R_{ij} = U_j^{\max},
  \end{align*}
  and similarly for the lower bound.
\end{remark}

\begin{remark}[About mass conservation.]\label{remark:conservation_low-order}
  Let $\bfn$ denote the unit normal vector to the boundary $\partial\Omega$.
  Assume the velocity field is divergence free $\nabla\cdot \bfv=0$ and that $\bfv\cdot \bfn=0$,
  then \eqref{low-order} is conservative in the sense that
  $\sum_i m_i U_i^L=\sum_i m_i U_i^n$. Since $\bfv\cdot \bfn=0$, via integration by parts,
  $K_{ij}=\int_\Omega (\bfv\cdot\nabla \phi_j)\phi_i d\bfx = -\int_\Omega\nabla\cdot(\bfv\phi_i)\phi_jd\bfx$. Therefore $\sum_j K_{ji}=-\sum_j\int_\Omega \nabla\cdot(\bfv\phi_j)\phi_i=0$
  (where we use the partition of unity property and
  the assumption that the velocity is divergence free).
  By construction $\sum_jD_{ij}^L=0\implies \sum_j D_{ij}^L U_j^n=\sum_jD_{ij}^L (U_j^n-U_i^n)$.
  Therefore,
  \begin{align*}
    \sum_j (K_{ij}+D_{ij}^L)U_j^n =
    \sum_j \left[(K_{ij}U_j^n - K_{ji}U_i^n) + D_{ij}^L(U_j^n-U_j^n)\right].
  \end{align*}
  Let $f_{ij}^L:=(K_{ij}U_j^n - K_{ji}U_i^n) + D_{ij}^L(U_j^n-U_j^n)$
  and note that (since $D^L_{ij}=D_{ji}^L$)
  $f_{ij}^L=-f_{ji}^L$; i.e., $f_{ij}^L$ forms a skew-symmetric matrix.
  Therefore, $\sum_i\sum_j f_{ij}^L=0\implies\sum_i m_i U_i^L=\sum_i m_i U_i^n$.
\end{remark}

\subsection{Modified RK methods with flux limiting on last stage}
Let $F_i(t)=\sum_j f_{ij}(t)$ be some high-order spatial discretization of
\eqref{transport_equation}
that doesn't necessarily satisfies positivity, local bounds or other monotonicity constraint
(like the one given by \eqref{local-bounds}).
%
We are interested on solving the semi-discretization
\begin{align}
  \frac{d}{dt} U_i(t)=:F_i(t) %-m_i^{-1}\sum_j(K_{ij}+D_{ij}^L)U_j(t)=:F_i(t)
\end{align}
via high-order Runge-Kutta methods. Moreover, we aim to modify the Runge-Kutta method
to impose the local bounds \eqref{local-bounds}.

\begin{remark}[About the spatial discretization.]
  In the following we consider a high-order discretization to be given by the pure
  continuous Galerkin scheme. This is off course highly oscillatory when the time discretization
  is given by a Runge-Kutta method and even unconditionally unstable for forward Euler.
  To remedy this problem, it is common in the literature of flux corrected transport
  for continuous Galerkin finite elements to pre-stabilize the high-order spatial discretization.
  See for example \cite{lohmann2017flux,guermond2014second}.
  %
  In addition, in this section we lump the mass matrix, 
  which introduces dispersive errors \cite{guermond2017effect}
  We remedy these two problems later.
  %
  Therefore, for now the high-order semi discretization of \eqref{transport_equation} is given by
  \begin{align}\label{high-order_semi-discretization}
    \frac{d}{dt}U_i^H(t)=F_i(t)=-m_i^{-1}\sum_j K_{ij}(t) U_j(t),
  \end{align}
  where $U^H$ denotes the high-order solution. See more details in \S\ref{sec:FE_discUpwind}.
\end{remark}

We apply a standard Runge-Kutta scheme to \eqref{high-order_semi-discretization} to get
\begin{align}\label{high-order_RK}
  U_i^H=U_i^n+\Delta t\sum_{m=1}^M b_k k_{m,i}.
\end{align}
Here we abuse the notation and let $U^H$ denote the high-order solution at time $t^{n+1}$.
The idea is to extract the forward Euler component from \eqref{high-order_RK}.
Since $k_{1,i} = F_i(t^n,U^n)=:F_i^n$, we get 
\begin{align}\label{last_stage_RK_aux1}
  U_i^H=U_i^n+\Delta t F^n_i + \Delta t (b_1-1)F^n_i + \Delta t \sum_{m=2}^M b_k k_{m,i}.
\end{align}
Moreover, we split the forward Euler flux as
\begin{align*}
  F_i^n=-m_i^{-1}\sum_j K^n_{ij}U_j^n = -m_i^{-1}\sum_j \left[(K^n_{ij}+D_{ij}^{L,n})U_j^n - D_{ij}^{L,n}U_j^n\right],
\end{align*}
where $D_{ij}^{L,n}$ is the discrete upwinding matrix
(given by \eqref{transport_and_discUpwinding_matrices}) at time $t^n$. 
Using the definition of the low-order method \eqref{low-order},
equation \eqref{last_stage_RK_aux1} becomes
\begin{align}\label{last_stage_RK_aux2}
  m_i(U_i^H-U_i^L) = \Delta t \sum_j D_{ij}^{L,n} U_j^n +
  \Delta t m_i (b_1-1)F^n_i + \Delta t m_i\sum_{m=2}^M b_k k_{m,i}.
\end{align}

Since $U^L$ satisfies the local bounds \eqref{local-bounds},
it is clear that the right hand side of \eqref{last_stage_RK_aux2}
is responsible not only for increasing the accuracy of $U^H$ in space and time
but also for creating violations of the local bounds.
%
We use the standard ideas in the Flux Corrected Transport (FCT) methodology
by \cite{boris1973flux} and \cite{zalesak1979fully}, see also \cite{kuzmin2012flux},
to limit the flux correction; i.e., the right hand side of \eqref{last_stage_RK_aux2}.
%
To do this we need to rewrite the right hand side of \eqref{last_stage_RK_aux2}
as an anti-diffusive flux correction $f_{ij}^C$.
To guarantee conservation of mass we need $f_{ij}^C=-f_{ji}^C$.
Let $K^m_{ij}$ be the entries of the transport operator corresponding to the $m$-th stage
in the RK method. Then 
\begin{align}
  k_{m,i} = F_{m,i}=\sum_j (K^m_{ij}U_j^m - K^m_{ji}U_i^m),
\end{align}
see remark \ref{remark:conservation_low-order} for more details.
Similarly $F_i^n=\sum_j(K^n_{ij}U_j^n - K^n_{ji}U_i^n)$.
Finally, since $\sum_j D_{ij}^{L,n}=0$, we get
$\sum_jD_{ij}^{L,n}U_j^n=\sum_jD_{ij}^{L,n}(U_j^n-U_i^n)$. Therefore, the flux correction is
given as follows: 
\begin{align}
  \begin{split}
  f_{ij}^C = \Delta t \sum_j D_{ij}^{L,n}(U_j^n-U_i^n)
  +  \Delta t m_i(b_1-1)\sum_j(K^n_{ij}U_j^n - K^n_{ji}U_i^n) 
  +  \Delta t m_i\sum_{m=2}^M b_k \sum_j (K^m_{ij}U_j^m - K^m_{ji}U_i^m).
  \end{split}
\end{align}
Since $D_{ij}^{L,n}=D_{ji}^{L,n}$, it is clear that $f_{ij}^C=-f_{ji}^C$.
Now we rewrite \eqref{last_stage_RK_aux2} as
\begin{align}
  m_i(U_i^H-U_i^L) = \sum_j f_{ij}^C.
\end{align}

Finally the solution at time $t^{n+1}$ is given by 
\begin{align}\label{MRK_last_stage}
  m_i(U_i^{n+1}-U_i^L)=\sum_j\mathcal{L}_{ij}f_{ij}^C,
\end{align}
where $\mathcal{L}_{ij}$ are the flux limiters given by
\begin{subequations}
\begin{align}
  \mathcal{L}_{ij}:=
  \begin{cases}
    \min(R_i^+,R_j^-), & \text{ if } f_{ij}^C\geq 0, \\
    \min(R_i^-,R_j^+), & \text{otherwise}, \\
  \end{cases}
\end{align}
where
\begin{align}
  R_i^+ &:=
  \begin{cases}
    \min\left[1, \frac{m_i(U_i^{\max}-U_i^L)}{f_i^{C,+}} \right], & f_i^{C,+}\neq 0, \\
    1, & \text{otherwise}
  \end{cases},
  & R_i^- &:=
  \begin{cases}
    \min\left[1, \frac{m_i(U_i^{\min}-U_i^L)}{f_i^{C,-}} \right], & f_i^{C,-}\neq 0, \\
    1, & \text{otherwise}
  \end{cases}, \\
  f_i^{C,+}&:=\sum_j \max(0,f_{ij}^C)  & f_i^{C,-}&:=\sum_j \min(0,f_{ij}^C).
\end{align}
\end{subequations}

\begin{remark}[About the discrete maximum principle (DMP).]
  The solution $U_i^{n+1}$ satisfies the following local bounds:
  \begin{align*}
    U_i^{\min} \leq U_i^{n+1} \leq U_i^{\max}.
  \end{align*}
  The proof is very standard (see for instance \cite{guermond2014second,kuzmin2012flux}).
  Assume $f_i^{C,+}\neq 0$, then 
  \begin{align*}
    \begin{split}
      m_i(U_i^{n+1}-U_i^L)
      &=\sum_j\mathcal{L}_{ij}f_{ij}^C
      \leq\sum_{j,f_{ij}^C\geq0}\mathcal{L}_{ij}f_{ij}^C
      =\sum_{j,f_{ij}^C\geq0}\min(R_i^+,R_j^-)f_{ij}^C 
      \leq R_i^+ f_i^{C,+}
      \leq \frac{m_i(U_i^{\max}-U_i^L)}{f_i^{C,+}}f_i^{C,+}\\
      &=m_i(U_i^{\max}-U_i^L)
      \implies U_i^{n+1}\leq U_i^{\max}. 
    \end{split}
  \end{align*}
  If $f_i^{C,+}=0$,
  \begin{align*}
    \begin{split}
      m_i(U_i^{n+1}-U_i^L)\leq R_i^+ f_i^{C,+}=0\implies U_i^{n+1}\leq U_i^L\leq U_i^{\max}.
    \end{split}
  \end{align*}
  The lower bound is proven similarly.
\end{remark}

\begin{remark}[About mass conservation.]
  The method \eqref{MRK_last_stage} is mass conservative in the sense that
  $\sum_i m_iU_i^{n+1}=\sum_i m_iU_i^n$.
  Recall that the limiters are symmetric and the flux correction is skew symmetric; i.e.,
  $\mathcal{L}_{ij}=\mathcal{L}_{ji}$ and $f_{ij}^C=-f_{ji}^C$, respectively.
  Therefore,
  \begin{align*}
    \begin{split}
    \mathcal{L}_{ij}f_{ij}^C=-\mathcal{L}_{ji}f_{ji}^C
    \implies \sum_i\sum_j \mathcal{L}_{ij}f_{ij}^C=0 
    \implies \sum_i m_iU_i^{n+1}=\sum_i m_iU_i^L.
    \end{split}
  \end{align*}
  Finally, due to remark \ref{remark:conservation_low-order}, $\sum_i m_iU_i^L=\sum_i m_iU_i^n$.
  Therefore, $\sum_i m_iU_i^{n+1}=\sum_i m_iU_i^n$.  
\end{remark}

\subsection{Modified RK methods with flux limiting on all stages}
\subsection{Hierarchical modified RK methods with flux limiting}


\bibliographystyle{plain}
%\bibliographystyle{abbrvnat}
\bibliography{References}


\end{document}
