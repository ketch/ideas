\documentclass[english,11pt]{article}

\usepackage{amsthm}
\usepackage[latin9]{inputenc}
\usepackage{babel}
\usepackage[hmargin=0.9in]{geometry}

%\usepackage{subfigure}
\usepackage{subfig}
\usepackage{graphicx}
\usepackage[colorlinks=true,citecolor=blue,urlcolor=blue]{hyperref}
\usepackage[numbers]{natbib}
%\usepackage{cite}
\let\cite=\citet

\usepackage{float}
\usepackage{url}
\usepackage{bm}

\usepackage{amsmath}
\usepackage{amssymb,amsfonts}
\usepackage{upgreek}
\usepackage[textsize=footnotesize]{todonotes}
\usepackage{stmaryrd}
%\newcommand{\todo}[1]{{\bf \color{red} #1}\vskip 10pt}

\newtheorem{remark}{Remark}
\numberwithin{remark}{subsection}

\makeatletter
\providecommand{\tabularnewline}{\\}
\makeatother

\graphicspath{{./figures/}}

% Macros -- use them
\newcommand{\bfu}{\mathbf{u}}
\newcommand{\bfv}{\mathbf{v}}
\newcommand{\bff}{\mathbf{f}}
\newcommand{\bfn}{\mathbf{n}}
\newcommand{\bfx}{\mathbf{x}}
\newcommand{\order}{{\mathcal O}}
\newcommand{\yhat}{\hat{y}}
\newcommand{\Khat}{\hat{K}}
\newcommand{\rhohat}{\hat{\rho}}
\newcommand{\etanot}{\eta_0}
\def\avg[#1]{\left\langle #1 \right\rangle}
\def\intavg[#1]{\llbracket #1 \rrbracket}
\def\ceff{c_\text{eff}}
\def\bk{\bf k}
\newcommand{\bx}{{\bf x}}

\begin{document}
\title{Polygonal hydraulic jumps: 3D free-surface flow simulations}
%\author{
%  Manuel Quezada de Luna \thanks{KAUST CEMSE, manuel.quezada@kaust.edu.sa.} \and
%  David I. Ketcheson \thanks{KAUST CEMSE, david.ketcheson@kaust.edu.sa.}}
\maketitle

%\begin{abstract}
%\end{abstract}

%\keywords{}

\section{Literature review}

\subsection{General remarks}
\begin{itemize}
\item The hydraulic jump and different theoretical and experimental aspects of it have been studied by many
  authors. For instance, \cite{rayleigh1914theory} studied the inviscid hydraulic jump and proposed an
  estimation of the radius of the jump. 
  \cite{watson1964radial}, used momentum balanced theory and consider inviscid and viscous flows to also propose estimates
  for the radius of the jump. These studies didn't include surface tension effects. 
  Later, \cite{bush2003influence} proposed a correction to the results by Watson to include the effects of surface
  tension on the radius of the jump. Many more models and corrections have been proposed to estimate the radius of the
  jump in different regimes and using different tools. 
  
\item According to \cite{labousse2013hydraulic}, \cite{bohr1996hydraulic} and \cite{watanabe2003integral}
  distinguised between hydraulic jumps of type I and II.
  \begin{itemize}
  \item[*] In type I there is only one vortex leading to the standard hydraulic jump.
  \item[*] In type II, a secondary vortex forms leading to the polygonal jump.
    Within type II, type IIa and IIb refer to polygonal jumps with one and two jumps respectively.
  \end{itemize}

\item The most up-to-date model for type II hydraulic jumps is given by \cite{martens2012model}.
\end{itemize}

\subsection{About the observation of instability}
\begin{itemize}
\item \cite{craik1981circular} was one of the first to observe the instability that leads to type II jumps.
  However, he used water and didn't observe the creation of polygons.
  He triggered the instability by increasing the downstream depth.

\item \cite{ishigai1977heat} observed and controlled the formation of the instability via the Froude number upstream.
  
\item According to \cite{liu1993hydraulic}, \cite{errico1986study} observed the formation of the instability
  by increasing the volume flow.

\item In \cite{liu1993hydraulic} the authors present more observations of the instability and remark that
  surface tension plays an important role. In particular, the authors describe the surface tension as a
  mechanism to stabilize the turbulent behavior of the instability. 
  They conclude that the height of the jump and the radius of the roller are critical for surface tension to act as
  a stabilization process. For exmple, in a hydraulic jump in a channel, the height of the jump and the radius of the roller
  are very large making the surface tension effects negligible. 

\item \cite{kasimov2008stationary} proposed a theory based on a shallow water model to determine the radius of the jump
  based on far-field and jump conditions including surface tension effects. The author showed that the steady circular
  jump can't exist if the surface tension is larger than a critical value.
  The author also remarked that the critical surface tension decreases as the viscosity increases.
  Furthermore, he suggests this might explain why the polygonal jumps appear only in highly viscous fluids. 
\end{itemize}


\subsection{Experimental observation of hydraulic jump of type IIa}
\begin{itemize}
\item In \cite{ellegaard1998creating}, the authors present and explain the phenomenon.
  Given a fluid (ethylene glycol) they can control the shape of the jump by adjusting the height of the external wall
  (which increases the height of the jump).
  At some point (as the height of the external wall increases), the polygonal shape turns into a circular shape again and eventually the jumps closes.
  If the height of the external wall is too small then a circular hydraulic jump (of type I) is formed.
  The authors provide the simple dimensionless parameter $\phi=\nu Q/g(h_\text{ext}^2-h_\text{int}^2)$, where $\nu$ is the viscosity
  and $Q$ is the flow rate. They observe that as $\phi$ increases, the number of corners increase.
  %
  The authors present details about the derivation of $\phi$ in \cite{ellegaard1999cover} and details about the experiment in
  \cite{ellegaard1996experimental}.

\item In \cite{bohr1996hydraulic}, the authors explain further the experimental results and start to attempt a numerical simulation.
  The model to be numerically solved is based on the steady state incompressible 2D-NS.
  The free surface is modeled by a linear and rigid boundary close to one observed in experiments.
  They neglect the effects of surface tension in this model. Even with this simplified model, the authors are able
  to produce the vortex in the type II jumps.
  Here it is important to remark that even without surface tension they observed (numerically) the formation of the
  roller (the second vortex in the type II jumps). Therefore, the apperance of the roller might not depend entirely upon the
  surface tension. 
  The authors also work out in this paper a simplified model to start to develop a theory.

\item In \cite{bush2006experimental}, experiments were performed to produce the polygonal hydraulic
  (type II) jump with many different shapes.
  The goal of the authors was to identify important dimensionless parameters that characterize type IIa jumps. They
  also present experimental results of type IIb jumps.
  %
  In this work, the authors demonstrated experimentally the importance of surface tension by adding a surfactant to the flow.
  A type IIa jump transformed to a type I when a surfactant was added (see figure 6).
  %
  In section 6, the authors explain a (potential) reason for which surface tension is important to create type II jumps.
  The idea is that with high surface tension, the instability (a pinch-off) is triggered by a lower surface energy configuration
  given by the polygonal shape.
  This process is characterized by the Ohnesorge number $Oh=\sigma R/(\mu\nu)$
  At high $Oh$ the pinch-off is resisted by fluid intertia and at low $Oh$ the pinch off is resisted by viscosity.
  %
  This work gives some results related to Ohnesorge number and frequency in the jump.
  This and other quantitative results in this work are something we might be able to validate.

\item \cite{andersen2010separation} remarks about the importance of the separation to create the roller in the type II jumps.
  
\end{itemize}

\subsection{About hydraulic jumps of type IIb}
\begin{itemize}
\item \cite{craik1981circular} observed the second jump using water right before the jump becomes unstable.
  It is important to remark that in this work the authors used water. It is believed that the viscosity and surface tension
  of water is not strong enough to create the polygonal shapes. Indeed, the authors in this work didn't observe those shapes.
  However, they observed the instability and, moreover, the formation of the second jump in the type IIb hydraulic jump
  (see figure 2b). 
  
\item \cite{bush2006experimental} has a good number of experiments with type IIb jumps with different shapes.

\item In \cite{liu1993hydraulic} the second jump was also observed (see figure 5).
  
\end{itemize}

\subsection{Other types of circular hydraulic jumps}
\begin{itemize}
\item{\bf Crown shape hydraulic jumps.} See figure 14 in \cite{bush2006experimental}.
  I have simulations of the SWEs with viscosity that seem to show a similar behavior.

\item
  {\bf Hydraulic bump.} In \cite{labousse2013hydraulic}, the authors introduce the idea of hydraulic bump,
  which happens below the hydraulic jump threshold. In this case the jet impinges a deep steady water and only 
  weakly perturbes the free surface creating a circular bump that can evolve into polygonal shapes
  (presumably due to the same reasons than with hydraulic jumps).

\end{itemize}

\subsection{Numerical simulations}

\begin{itemize}
\item In \cite{bohr1996hydraulic}, the authors assumed a fixed profile for the free surface based on experimental data.
  Then they solved the steady state 2D-NS equations and observed the two vortices.

\item In \cite{yokoi2002mechanism}, numerical simulations of the compressible NS equations are performed.
  They use axisymmetrical simulations (i.e. 2D) and consider type I and type II jumps.
  They concluded that the transition between type I to II jumps is heavily affected by surface tension.
  %
  It is important to remark that the authors marked a jump as type II if the roller is observed.
  Off course with 2D simulations it is not possible to observe the polygonal shapes. 
  %
  The numerical methods are based on a (non conservative) level-set approach.
  The results in this paper have been presented with more physics oriented details in
  \cite{yokoi2000relationships,yokoi1999numerical}.
  At this point (in time) the authors claimed that there is no model for polygonal hydraulic jumps.

\item In \cite{passandideh2011numerical} the authors solved the incompressible Navier-Stokes equations for two immiscible
  viscous flows coupled with a volume-of-fluid approach to track the free surface.
  The authors considered type I and II jumps.
  %
  It appears to me that only two-dimensional simulations are performed; however, the authors extruded
  the solution in the angular direction to obtain a three-dimensional visualization. Off course this is
  only valid if the jump is axisymmetric. Therefore, the authors didn't obtain polygonal jumps.
\end{itemize}

\section{Objectives}
\begin{itemize}
\item Simulate for the first time the 3D hydraulic jump of type II to observe the polygonal shapes.
\item If possible try to obtain other type of jumps like type IIb, crown hydraulic jumps and hydraulic bumps (of type I and II). 
\item After proper validation against experimental data, extend the experiments to other regimes that might be difficult
  to achieve in a laboratory. For example, highly viscous but without surface tension or vice versa,
  see the effect of slip BCs on the bottom of the tank (is friction necessary to have polygonal shapes?).
\item Confirm the role and importance on surface tension in the creation the polygonal shapes.
\item Test Kasimov's hypothesis that higher viscous effects might reduce the importance of surface tension in the
  creation of the polygonal jump.
\item As byproduct simulate (for the first time) the 3D hydraulic jump of type I.
\end{itemize}

\section{Two-dimensional simulations}

\section{Three-dimensional simulations}

%\bibliographystyle{plain}
\bibliographystyle{unsrtnat}
%\bibliographystyle{abbrvnat}
\bibliography{ref}

\end{document}

